% Chapter Template

\chapter{IMPLEMENTATION AND TESTING} % Main chapter title

\label{Chapter5} % Change X to a consecutive number; for referencing this chapter elsewhere, use \ref{ChapterX}

\lhead{Chapter 5. \emph{IMPLEMENTATION AND TESTING}} % Change X to a consecutive number; this is for the header on each page - perhaps a shortened title

%----------------------------------------------------------------------------------------
%	SECTION 1
%----------------------------------------------------------------------------------------

\section{Implementation Approaches}
Define the plan of implementation, and the standards you have used in the implementation.
\section{Coding Details and Code Efficiency}
Students not need include full source code, instead, include only the important codes (algorithms, applets code, forms code etc).The program code should contain comments needed for explaining the work a piece of code does. Comments may be needed to explain why it does it, or, why it does a particular way. You can explain the function of the code with a shot of the output screen of that program code.

\textit{Code Efficiency: You should explain how your code is efficient and how you have handled code optimisation.}

\subsection{Code Efficiency}

\section{Testing Approach}
Testing should be according to the scheme presented in the system design chapter and should follow some suitable model – e.g., category partition, state machine-based. Both functional testing and user-acceptance testing are appropriate. Explain your approach of testing.

\textbf{Project Structure}

\textit{Unit Testing: Unit testing deals with testing a unit or module as a whole. This would test the interaction of many functions but, do confine the test within one module.}

\textit{Integrated Testing: Brings all the modules together into a special testing environment, then checks for errors, bugs and interoperability. It deals with tests for the entire application. Application limits and features are tested here.}

\subsection{Unit Testing}

\subsection{Integrated Testing}

\section{Modifications and Improvements}
Once you finish the testing you are bound to be faced with bugs, errors and you will need to modify your source code to improve the system. Define what modification you implemented in the system and how it improved your system.
